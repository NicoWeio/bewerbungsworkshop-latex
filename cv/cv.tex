\documentclass[a4paper, 12pt]{moderncv}

% see texdoc moderncv for options
\moderncvtheme{casual}

\usepackage[margin=1.5cm, bottom=2.5cm]{geometry}
\recomputelengths


% font setup
\usepackage{fontspec}
\setromanfont{Libertinus Serif}
\setsansfont{Libertinus Sans}

% use sans-serif font as default font in the document
\renewcommand\familydefault\sfdefault%


\usepackage{microtype}

\usepackage[ngerman]{babel}

\usepackage{graphicx}

% for publication list
\usepackage[autostyle]{csquotes}
\usepackage[sorting=ydnt]{biblatex}
\addbibresource{publications.bib}

% add note below main citation, a little smaller.
% Helpful to summarize the paper / thesis and to explain main outcomes
\DeclareSourcemap{
  \maps[datatype=bibtex]{
    \map{
      \step[fieldsource=note, final]
      \step[fieldset=addendum, origfieldval, final]
      \step[fieldset=note, null]
    }
  }
}
\DeclareFieldFormat{addendum}{%
  \newline
  \small%
  {\color{darkgray}{#1}}
}

% u.a. -> et al.
\DefineBibliographyStrings{german}{andothers = {{et\,al\adddot}}}



\firstname{{\fontsize{24}{24}\selectfont Dr.~rer.~nat.}\\Max}
\familyname{Mustermann}
\photo{example-image-a}
\title{Lebenslauf}


\address{Otto-Hahn-Str. 4a}{44227 Dortmund}
\mobile{+49 123 456789}
\email{max10.mustermann@tu-dortmund.de}

\usepackage{xpatch}

% do not justify text
\xpatchcmd{\cvitem}{#3}{\raggedright #3\par}{}{}
\xpatchcmd{\cventry}{\small}{\small\raggedright}{}{}

% remove dot after cventry
\xpatchcmd{\cventry}{.\strut}{\strut}{}{}


\usepackage{xcolor}
\colorlet{color1}{red!70!black}

\setlength{\hintscolumnwidth}{4cm}
\setlength{\footskip}{40pt}

\usepackage[unicode, colorlinks=true, urlcolor=color1, linkcolor=color1]{hyperref}


\begin{document}
\raggedright

\maketitle
\setlength{\parskip}{-1.0ex}

\section{Persönliche Daten}

\cvline{Geburtsdatum}{1984-02-30}
\cvline{Familienstand}{ledig}
\cvline{Staatsangehörigkeit}{deutsch}

\section{Ausbildung}

\cventry{20xx–20yy}{Abitur}{Tolles Gymnasium}{Heimatstadt}{Note:~4,0}{}
\cventry{20yy–20zz}{Bachelor of Science Physik}{TU~Dortmund}{}{Note:~4,0}{}
\cventry{20zz–20aa}{Master of Science Physik}{TU~Dortmund}{}{Note:~4,0}{}
\cventry{20aa–20bb}{Promotion}{TU~Dortmund}{}{Note:~rite}{}

\section{Beruf}
\cventry{20aa–20bb}{Wissenschaftlicher Mitarbeiter}{AG Einstein}{TU Dortmund}{}{Kurze Beschreibung}

\section{Lehre}
\cventry{20aa-20bb}{Einführung in die Quatschphysik}{TU Dortmund}{}{}{Tolle Vorlesung}
\cventry{2014-2022}{PeP et al. Toolbox Workshop}{TU Dortmund}{}{}{Kurs für Physikstudierende. Einführung in wissenschaftliches Programmieren und Datenanalyse mit Python, Git, Make und Verfassen wissenschaftlicher Texte mit \LaTeX{}.}

\newpage
\section{Interessen}
\cvlistdoubleitem{Photographie}{Kochen}
\cvlistdoubleitem{Fechten}{Programmieren}

\section{Ehrenamtliches Engagement}
\cventry{2012–heute}{Vorstandsmitglied}{Physikstudierende und ehemalige Physikstudierende der TU Dortmund et al. e.\,V}{}{}{Organisation des Toolbox-Workshops, der Sommerakademie, Absolventenfeier und von weiteren Workshops und Hackathons. Betreuung der IT-Infrastruktur.}

\section{Programmiersprachen (Auswahl)}
\cvlanguage{Python}{Experte}{Maintainer von Bibliothek X, Beiträge zu Open-Source Projekten Y und Z (siehe Software-Projekte). Lehrender bei Kursen und Workshops.}
\cvlanguage{\LaTeX}{Experte}{Langjährige Leitung des \LaTeX{}-Kurses von PeP et al.}
\cvlanguage{C++}{Experte}{Habe \enquote{Effective Modern C++} gelesen}
\cvlanguage{Rust, Java, Julia, JavaScript, Go}{Grundkenntnisse}{}

\section{Veröffentlichungen (Auswahl)}
\nocite{*}
\printbibliography[heading=none]

\section{Software-Projekte (Auswahl)}
\cvline{Toolbox-Workshop}{\href{https://github.com/pep-dortmund/toolbox-workshop}{github.com/pep-dortmund/toolbox-workshop}}

\end{document}
